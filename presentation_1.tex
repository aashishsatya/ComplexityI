%%%%%%%%%%%%%%%%%%%%%%%%%%%%%%%%%%%%%%%%%
% Beamer Presentation
% LaTeX Template
% Version 1.0 (10/11/12)
%
% This template has been downloaded from:
% http://www.LaTeXTemplates.com
%
% License:
% CC BY-NC-SA 3.0 (http://creativecommons.org/licenses/by-nc-sa/3.0/)
%
%%%%%%%%%%%%%%%%%%%%%%%%%%%%%%%%%%%%%%%%%

%----------------------------------------------------------------------------------------
%	PACKAGES AND THEMES
%----------------------------------------------------------------------------------------

\documentclass{beamer}

\mode<presentation> {

% The Beamer class comes with a number of default slide themes
% which change the colors and layouts of slides. Below this is a list
% of all the themes, uncomment each in turn to see what they look like.

\usetheme{default}
%\usetheme{AnnArbor}
%\usetheme{Antibes}
%\usetheme{Bergen}
%\usetheme{Berkeley}
%\usetheme{Berlin}
%\usetheme{Boadilla}
%\usetheme{CambridgeUS}
%\usetheme{Copenhagen}
%\usetheme{Darmstadt}
%\usetheme{Dresden}
%\usetheme{Frankfurt}
%\usetheme{Goettingen}
%\usetheme{Hannover}
%\usetheme{Ilmenau}
%\usetheme{JuanLesPins}
%\usetheme{Luebeck}
%\usetheme{Madrid}
%\usetheme{Malmoe}
%\usetheme{Marburg}
%\usetheme{Montpellier}
%\usetheme{PaloAlto}
%\usetheme{Pittsburgh}
%\usetheme{Rochester}
%\usetheme{Singapore}
%\usetheme{Szeged}
%\usetheme{Warsaw}

% As well as themes, the Beamer class has a number of color themes
% for any slide theme. Uncomment each of these in turn to see how it
% changes the colors of your current slide theme.

%\usecolortheme{albatross}
%\usecolortheme{beaver}
%\usecolortheme{beetle}
%\usecolortheme{crane}
%\usecolortheme{dolphin}
%\usecolortheme{dove}
%\usecolortheme{fly}
%\usecolortheme{lily}
%\usecolortheme{orchid}
%\usecolortheme{rose}
%\usecolortheme{seagull}
%\usecolortheme{seahorse}
%\usecolortheme{whale}
%\usecolortheme{wolverine}

%\setbeamertemplate{footline} % To remove the footer line in all slides uncomment this line
%\setbeamertemplate{footline}[page number] % To replace the footer line in all slides with a simple slide count uncomment this line

%\setbeamertemplate{navigation symbols}{} % To remove the navigation symbols from the bottom of all slides uncomment this line
}

\usepackage{graphicx} % Allows including images
\usepackage{booktabs} % Allows the use of \toprule, \midrule and \bottomrule in tables
%\usepackage[document]{ragged2e} % for center align

%----------------------------------------------------------------------------------------
%	TITLE PAGE
%----------------------------------------------------------------------------------------

\title[Complexity I]{Computational Complexity I} % The short title appears at the bottom of every slide, the full title is only on the title page

\author{Aashish Satyajith} % Your name
\institute[CMI] % Your institution as it will appear on the bottom of every slide, may be shorthand to save space
{
Chennai Mathematical Institute \\ % Your institution for the title page
\medskip
\textit{aashishsatyajith@gmail.com} % Your email address
}
\date{\today} % Date, can be changed to a custom date

\begin{document}

\begin{frame}
\titlepage % Print the title page as the first slide
\end{frame}

% \begin{frame}
% \frametitle{Overview} % Table of contents slide, comment this block out to remove it
% \tableofcontents % Throughout your presentation, if you choose to use \section{} and \subsection{} commands, these will automatically be printed on this slide as an overview of your presentation
% \end{frame}

%----------------------------------------------------------------------------------------
%	PRESENTATION SLIDES
%----------------------------------------------------------------------------------------

%------------------------------------------------
% \section{First Section} % Sections can be created in order to organize your presentation into discrete blocks, all sections and subsections are automatically printed in the table of contents as an overview of the talk
% %------------------------------------------------

% \subsection{Subsection Example} % A subsection can be created just before a set of slides with a common theme to further break down your presentation into chunks

% \begin{frame}
% \frametitle{Paragraphs of Text}
% Sed iaculis dapibus gravida. Morbi sed tortor erat, nec interdum arcu. Sed id lorem lectus. Quisque viverra augue id sem ornare non aliquam nibh tristique. Aenean in ligula nisl. Nulla sed tellus ipsum. Donec vestibulum ligula non lorem vulputate fermentum accumsan neque mollis.\\~\\

% Sed diam enim, sagittis nec condimentum sit amet, ullamcorper sit amet libero. Aliquam vel dui orci, a porta odio. Nullam id suscipit ipsum. Aenean lobortis commodo sem, ut commodo leo gravida vitae. Pellentesque vehicula ante iaculis arcu pretium rutrum eget sit amet purus. Integer ornare nulla quis neque ultrices lobortis. Vestibulum ultrices tincidunt libero, quis commodo erat ullamcorper id.
% \end{frame}

% %------------------------------------------------

% \begin{frame}
% \frametitle{Bullet Points}
% \begin{itemize}
% \item Lorem ipsum dolor sit amet, consectetur adipiscing elit
% \item Aliquam blandit faucibus nisi, sit amet dapibus enim tempus eu
% \item Nulla commodo, erat quis gravida posuere, elit lacus lobortis est, quis porttitor odio mauris at libero
% \item Nam cursus est eget velit posuere pellentesque
% \item Vestibulum faucibus velit a augue condimentum quis convallis nulla gravida
% \end{itemize}
% \end{frame}

% %------------------------------------------------

% \begin{frame}
% \frametitle{Blocks of Highlighted Text}
% \begin{block}{Block 1}
% Lorem ipsum dolor sit amet, consectetur adipiscing elit. Integer lectus nisl, ultricies in feugiat rutrum, porttitor sit amet augue. Aliquam ut tortor mauris. Sed volutpat ante purus, quis accumsan dolor.
% \end{block}

% \begin{block}{Block 2}
% Pellentesque sed tellus purus. Class aptent taciti sociosqu ad litora torquent per conubia nostra, per inceptos himenaeos. Vestibulum quis magna at risus dictum tempor eu vitae velit.
% \end{block}

% \begin{block}{Block 3}
% Suspendisse tincidunt sagittis gravida. Curabitur condimentum, enim sed venenatis rutrum, ipsum neque consectetur orci, sed blandit justo nisi ac lacus.
% \end{block}
% \end{frame}

% %------------------------------------------------

% \begin{frame}
% \frametitle{Multiple Columns}
% \begin{columns}[c] % The "c" option specifies centered vertical alignment while the "t" option is used for top vertical alignment

% \column{.45\textwidth} % Left column and width
% \textbf{Heading}
% \begin{enumerate}
% \item Statement
% \item Explanation
% \item Example
% \end{enumerate}

% \column{.5\textwidth} % Right column and width
% Lorem ipsum dolor sit amet, consectetur adipiscing elit. Integer lectus nisl, ultricies in feugiat rutrum, porttitor sit amet augue. Aliquam ut tortor mauris. Sed volutpat ante purus, quis accumsan dolor.

% \end{columns}
% \end{frame}

% %------------------------------------------------
% \section{Second Section}
% %------------------------------------------------

% \begin{frame}
% \frametitle{Table}
% \begin{table}
% \begin{tabular}{l l l}
% \toprule
% \textbf{Treatments} & \textbf{Response 1} & \textbf{Response 2}\\
% \midrule
% Treatment 1 & 0.0003262 & 0.562 \\
% Treatment 2 & 0.0015681 & 0.910 \\
% Treatment 3 & 0.0009271 & 0.296 \\
% \bottomrule
% \end{tabular}
% \caption{Table caption}
% \end{table}
% \end{frame}

% %------------------------------------------------

% \section{Introduction}

% \begin{frame}
% \frametitle{Theorem}
% \begin{theorem}[Mass--energy equivalence]
% $E = mc^2$
% \end{theorem}
% \end{frame}

% %------------------------------------------------

% \begin{frame}[fragile] % Need to use the fragile option when verbatim is used in the slide
% \frametitle{Verbatim}
% \begin{example}[Theorem Slide Code]
% \begin{verbatim}
% \begin{frame}
% \frametitle{Theorem}
% \begin{theorem}[Mass--energy equivalence]
% $E = mc^2$
% \end{theorem}
% \end{frame}\end{verbatim}
% \end{example}
% \end{frame}

% %------------------------------------------------

% \begin{frame}
% \frametitle{Figure}
% Uncomment the code on this slide to include your own image from the same directory as the template .TeX file.
% %\begin{figure}
% %\includegraphics[width=0.8\linewidth]{test}
% %\end{figure}
% \end{frame}

% %------------------------------------------------

% \begin{frame}[fragile] % Need to use the fragile option when verbatim is used in the slide
% \frametitle{Citation}
% An example of the \verb|\cite| command to cite within the presentation:\\~

% This statement requires citation \cite{p1}.
% \end{frame}

% %------------------------------------------------

% \begin{frame}
% \frametitle{References}
% \footnotesize{
% \begin{thebibliography}{99} % Beamer does not support BibTeX so references must be inserted manually as below
% \bibitem[Smith, 2012]{p1} John Smith (2012)
% \newblock Title of the publication
% \newblock \emph{Journal Name} 12(3), 45 -- 678.
% \end{thebibliography}
% }
% \end{frame}

% %------------------------------------------------

\section{Definitions}

\begin{frame}

\frametitle{Definitions}

\begin{itemize}
\item A \textbf{universal Turing machine} is a TM $\mathcal{U}$ such that for every $x, \alpha \in \{0,1\}^*, \mathcal{U}(x, \alpha) = M_{\alpha}(x)$, where $M_{\alpha}$ denotes the TM represented by $\alpha$.
\item A function $T: \mathbb{N} \to \mathbb{N}$ is \textbf{time constructible} if $T(n) \geq n$ and there is a TM $M$ that computes the function $x \to T(x)$ in time $T(n)$

\end{itemize}

\end{frame}

\begin{frame}

\frametitle{DTIME, NTIME, and P}

\begin{itemize}
\item Let $T: \mathbb{N} \to \mathbb{N}$ be some function. A language $L$ is in $DTIME(T(n))$ iff there is a TM that runs in time $c \cdot T(n)$ for some constant $c > 0$ and decides $L$.
\item If the TM is non-deterministic, the language $L$ is in $NTIME(T(n))$
\end{itemize}

\begin{block}{The class P}
$P = \cup_{c \geq 1} DTIME(n^c)$
\end{block}


\end{frame}

\begin{frame}

\frametitle{The class NP}

A language $L \subseteq \{0,1\}^*$ is in $NP$ if there exists a polynomial $p: \mathbb{N} \to \mathbb{N}$ polynomial-time TM $M$ (called the verifier for $L$) such that $\forall x \in \{0,1\}^*$

\begin{center}
$x \in L \iff \exists u \in \{0,1\}^{p(|x|)}\ \textrm{s.t.} M(x,u) = 1$
\end{center}

...in which case $u$ is called a \textbf{certificate} for $x$.\\
Also, $NP = \cup_{c \geq 1} NTIME(n^c)$

\end{frame}

\begin{frame}

\frametitle{Some More Classes}

\begin{itemize}
\item $coNP = \{\overline{L}: L \in NP\}$
\item $EXP = \cup_{c \geq 1} DTIME(2^{n^c})$
\end{itemize}

\end{frame}



\section{Hierarchy Theorems}

\subsection{Space Hierarchy Theorems}

\subsection{Time Hierarchy Theorems}

\subsection{Ladner's Theorem}

\begin{frame}
\frametitle{Ladner's Theorem}

\begin{block}{Ladner's Theorem}
Suppose that $P \neq NP$. Then there exists a language $L \in NP \backslash P$ that is not $NP$-complete.
\end{block}

\end{frame}

\subsection{Baker -- Gill -- Solovay Theorem}

\begin{frame}
\frametitle{Baker -- Gill -- Solovay Theorem}

An \textbf{oracle Turing machine} is a TM $M$ with "oracle" $O \subseteq \{0,1\}^*$ (written $M^O$) and has free access to membership queries in $O$.

\begin{block}{Baker -- Gill -- Solovay Theorem}
There exists oracles $A, B$ such that $P^A = NP^A$ and $P^B \neq NP^B$
\end{block}

For proof:

\begin{itemize}
\item Set $A$ as $TQBF$
\item Use diagonalization to get oracle $B$
\end{itemize}

\end{frame}

\subsection{Savitch's Theorem}

\begin{frame}
\frametitle{Savitch's Theorem}

\begin{block}{Savitch's Theorem}
For any space constructible $S: \mathbb{N} \to \mathbb{N}$ with $S(n) \geq log\ n$, $NSPACE(S(n)) \subseteq SPACE(S(n)^2)$
\end{block}

(Proof using configuration graphs and trying to reach intermediate vertices)

\end{frame}

\subsection{Immerman -- Szelepscenyi Theorem}

\begin{frame}
\frametitle{Immerman -- Szelepscenyi Theorem}

\begin{block}{Immerman -- Szelepscenyi Theorem}
$NL = coNL$
\end{block}

For proof:

\begin{itemize}
\item Show that $\overline{PATH} \in NL$
\item $\overline{PATH}$ is $coNL$ complete and so $NL = coNL$
\end{itemize}

\end{frame}

\section{Polynomial Hierarchy}

\begin{frame}
\frametitle{Polynomial Hierarchy}

For $i \geq 1$ a language $L$ is in $\sum_i^p$ if there exists a polynomial-time TM $M$ and a polynomial $q$ such that

\begin{center}
$x \in L \iff \exists u_1 \in \{0,1\}^{q(|x|)}\ \forall u_2 \in \{0,1\}^{q(|x|)}...Q_iu_i \in \{0,1\}^{q(|x|)}\ M(x,u_1,...,u_i) = 1$
\end{center}

where $Q_i$ denotes $\forall$ or $\exists$ depending on whether $i$ is even or odd respectively.\\
The \textbf{polynomial hierarchy} is the set $PH = \cup_i \sum_i^p$.\\
Also for every $i$, define $\Pi_i^p = co\sum_i^p = \{\overline{L}: L \in \sum_i^p\}$\\
In terms of oracles: For every $i \geq 2,\sum_i^p = NP^{\sum_{i-1} SAT}$ 

\end{frame}

\section{Boolean Circuits}

\begin{frame}

\frametitle{Boolean Circuits}

\begin{itemize}
\item For every $n \in \mathbb{N}$, an $n$-input, single output \textbf{Boolean circuit} is a directed acyclic graph with $n$ sources and one sink. All non-source vertices are called gates and are labeled with one of $\lor, \land$ or $\neg$.
\item Vertices labeled with OR and AND have fan in equal to two, one for all else
\item The \textbf{size} of $C$, denoted by $|C|$, is the number of vertices in it.
\end{itemize}

\end{frame}

\begin{frame}

\frametitle{Circuit Families}

\begin{itemize}
\item Let $T: \mathbb{N} \to \mathbb{N}$ be a function. A $T(n)$-size circuit family is a sequence $\{C_n\}_{n \in \mathbb{N}}$ of Boolean circuits, where $C_n$ has $n$ inputs and a single output, and its size $|C_n| \leq T(n)$ for every $n$.
\item We say that a language $L$ is in $SIZE(T(n))$ if there exists a $T(n)$-size circuit family $\{C_n\}_{n \in \mathbb{N}}$ such that for every $x \in \{0,1\}^n$, $x \in L \iff C_n(x) = 1$.
\end{itemize}

\end{frame}

\begin{frame}

\frametitle{The class $P_{/poly}$}

\begin{block}{The class $P_{/poly}$}
$P_{/poly}$ is the class of languages that are determined by polynomial-sized circuit families, i.e. $P_{/poly} = \cup_{c} SIZE(n^c)$
\end{block}

\begin{itemize}
\item Theorem: $P \subseteq P_{/poly}$
\begin{itemize}
\item  Proof: Show that for every oblivious $T(n)$-time TM $M$, there exists an $O(T(n))$-sized circuit family $\{C_n\}_{n \in \mathbb{N}}$
\end{itemize}
\item Theorem: $P \subset P_{/poly}$
\begin{itemize}
\item  Proof: UHALT = \{$1^n$: $n$'s binary expansion encodes a pair $(M, x)$ such that M halts on input $x$\} is in $P_{/poly}$ but not in $P$.
\end{itemize}
\end{itemize}

\end{frame}

\begin{frame}

\frametitle{Uniformly Generated Circuits}

\begin{itemize}
\item A circuit family $\{C_n\}$ is \textbf{P-uniform} if there is a polynomial time TM that on input $1^n$ outputs the description of the circuit $C_n$.
\end{itemize}

\end{frame}

\section{Randomized Computation}

\subsection{Some Results}

\begin{frame}
\frametitle{Some Results}
\begin{itemize}
\item $BPP \subseteq P_{/poly}$
\end{itemize}
\end{frame}

\subsubsection{Sipser--Gacs Theorem}

\begin{frame}
\frametitle{Sipser--Gacs Theorem}
\begin{block}{Sipser--Gacs Theorem}
$BPP \subseteq \sum_2^p \cap\ \Pi_2^p$
\end{block}

Proof idea: Shifting tiles!

\end{frame}

\section{Interactive Proofs}

\begin{frame}
\frametitle{Probabilistic Verifiers and the class IP}

For an integer $k \geq 1$ (that may depend upon the input length), we say that a language $L$ is in $IP[k]$ if there is a probabilistic polynomial time TM $V$ that can have a $k$-round interaction with a function $P:\{0,1\} \to \{0,1\}^*$ such that\\

(Completeness) $x \in L \Rightarrow \exists P\ Pr[\textrm{out}_V \langle V, P \rangle (x) = 1] \geq 2/3$\\
(Soundness) $x \notin L \Rightarrow \forall P\ Pr[\textrm{out}_V \langle V, P \rangle (x) = 1] \leq 1/3$\\

where all probabilities are over the choice of random tosses $r \in_R \{0,1\}^m$ and $\textrm{out}_f$ denotes the output of $f$ at the end of the "interaction".\\

Now define

\begin{block}{The Class IP}
$IP = \cup_{c \geq 1} IP[n^c]$
\end{block}

\end{frame}

\begin{frame}
\frametitle{AM and MA}

\begin{itemize}
\item For every $k$ the complexity class $AM[k]$ is defined as the subset of $IP[k]$ obtained when we restrict the verifier's messages to be random bits, and not allowing it to use any other random bits that are not contained in these messages.
\item \textbf{AM = AM[2]} (note the distinction from class IP!)
\item The class \textbf{MA} denotes the class of languages with a two round public-coin interactive roof with the prover sending the first message. 
\end{itemize}




\end{frame}

% Linear equations can be typed in as:
% \begin{equation*}
% \begin{array}{ll@{}ll}
% \text{minimize}  & \displaystyle\sum\limits_{j=1}^{m} w_{j}&x_{j} &\\
% \text{subject to}& \displaystyle\sum\limits_{j:e_{i} \in S_{j}}   &x_{j} \geq 1,  &i=1 ,..., n\\
%                  &                                                &x_{j} \in \{0,1\}, &j=1 ,..., m
% \end{array}
% \end{equation*}

%------------------------------------------------

\begin{frame}
\Huge{\centerline{The End}}
\end{frame}

%----------------------------------------------------------------------------------------

\end{document}